\documentclass{article}

\usepackage{fancyhdr}
\usepackage{extramarks}
\usepackage{amsmath}
\usepackage{amsthm}
\usepackage{amsfonts}
\usepackage{tikz}
\usepackage[plain]{algorithm}
\usepackage{algpseudocode}
\usepackage{graphicx}

\usetikzlibrary{automata,positioning}

%
% Basic Document Settings
%

\topmargin=-0.45in
\evensidemargin=0in
\oddsidemargin=0in
\textwidth=6.5in
\textheight=9.0in
\headsep=0.25in

\linespread{1.1}

\pagestyle{fancy}
\lhead{Yousef Alaa Awad}
\chead{\hmwkClass\: \hmwkTitle}
\rhead{\firstxmark}
\lfoot{\lastxmark}
\cfoot{\thepage}

\renewcommand\headrulewidth{0.4pt}
\renewcommand\footrulewidth{0.4pt}

\setlength\parindent{0pt}

%
% Create Problem Sections
%

\setcounter{secnumdepth}{0}
\newcounter{partCounter}
\newcounter{homeworkProblemCounter}
\setcounter{homeworkProblemCounter}{1}

\newcommand{\hmwkTitle}{Homework\ \#1}
\newcommand{\hmwkDueDate}{February 2, 2026}
\newcommand{\hmwkClass}{Computer Communication Networks}

%
% Title Page
%

\title{
    \vspace{2in}
    \textmd{\textbf{\hmwkClass:\ \hmwkTitle}}\\
    \normalsize\vspace{0.1in}
    \vspace{3in}
}

\author{Yousef Alaa Awad}

% Problems start here
\begin{document}

\maketitle
\pagebreak

\section{Q1}
Suppose users share a 6 Mbps link, and the total number of users is 4. Also suppose each user transmits continuously at 2 Mbps when transmitting, but each user transmits only 40 percent of the time. (See the discussion of statistical multiplexing in Section 1.3.)

\subsection{A) When circuit switching is used, how many users can be supported?}
$$ n_{max} = \frac{B_{out}}{B_{in}} = \frac{6Mbps}{2Mbps} = 3\ Users $$

\subsection{B) Find the probability that a given user is transmitting.}
$$ p_{not\ 0} = \binom{4}{0}(0.4)^0(1-0.4)^{4-0} = 0.216 \rightarrow p_{not\ 0} = 1 - 0.1296 = 0.8704 $$

\subsection{C) Find the probability that only one of the four users is transmitting.}
$$ p_1 = \binom{4}{1}*(0.4)^1(1-0.4)^{4-1} = 0.3456 $$

\subsection{D) Find the probability that at any given time, all four users are transmitting simultaneously. Find the fraction of time during which the queue grows.}
$$ p_{all} = \binom{4}{4}*(0.4)^4(1-0.4)^{4-4} = 0.0256 $$
Now, the queue grows when there are more active users than there is possible active links, therefore, the fraction of time probability is the same as P(4) which is 2.56\%.

\section{Q2}

\subsection{A) How long does it take a packet of length 3,000 bytes to propagate over a link of distance 3,5000 km, propagation speed 2.4*108 m/s, and transmission rate 16 Mbps?}
$$ d_{prop} = \frac{distance}{speed} = \frac{35,000}{2.4*10^8m/s} \approx 0.1458s \approx 145.8ms $$

\subsection{B) More generally, how long does it take a packet of length L to propagate over a link of distance d, propagation speed s, and transmission rate Rbps?}
The general form of the propogation delay is:
$$ d_{prop} = \frac{distance}{speed} $$

\subsection{C) Does this delay depend on packet length?}
The delay does not depend on the packet length. Delay is a function of only distance and speed and not the length.

\subsection{D) Does this delay depend on transmission rate?}
No. The transmission rate does not effect the delay whatsoever since it is, again, only based on the speed of the physical medium and the length between the 2 devices.

\section{Q3}
\textbf{Suppose Host A wants to send a large file to Host B. The path from Host A to Host B has three links, of rates R1=800 kbps, R2=4 Mbps, and R3=1 Mbps.}

\subsection{A) Assuming no other traffic in the network, what is the throughput for the file transfer?}
Throughput is always limited by the slowest link therefore the throughput 800kpbs.

\subsection{B) Suppose the file is 4 million bytes. Dividing the file size by the throughput, roughly how long will it take to transfer the file to Host B?}
If a file is 4 million bytes, then it is 32 million bits. The time it takes is then the following algebra:
$$ Time = \frac{Size}{Throughput} = \frac{32*10^6}{800*10^3} = 40\ seconds $$

\subsection{C) Repeat (A) and (B), but now with R2 reduced to 600 kbps.}
When reduced to 600 kbps, the time taken is now...
$$ Time = \frac{Size}{Throughput} = \frac{32*10^6}{600*10^3} \approx 53.33\ seconds $$


\section{Q4}

\subsection{A) Which layers in the Internet protocol stack does a router process?}
The protocol layers used in a router is the Physical, Link, and Network layer.

\subsection{B) Which layers does a link-layer switch process?}
The layers that the switch uses are the Link and Physical layer.

\subsection{C) Which layers does a host process?}
The host covers all 5 layers, that being the Application, Transport, Network, Link, and Physical layers.

\section{Q5}
\textbf{Equation $d_{end-to-end}=\frac{N*L}{R}$ (Equation 1.1 in textbook) gives a formula for the end-to-end delay of sending one packet of length L over N links of transmission rate R. Generalize this formula for sending P such packets back-to-back over the N links.}
$$ d_{end-to-end} = (N+P-1)*\frac{L}{R} $$

\section{Q6}
\textbf{Consider the queuing delay in a router buffer. Let I denote traffic intensity, that is, $I=\frac{L*a}{R}$. Suppose that the queuing delay takes the form $\frac{I*L}{R*(1-I)}$ for $I < 1$.}

\subsection{A) Provide a formula for the total delay, that is, the queuing delay plus the transmission delay.}
$$ d_{total} = d_{queueing} + q_{trans} = \frac{I*L}{R*(1-I)}+\frac{L}{R} = \frac{L}{R}*(\frac{I}{1-I}+1) = \frac{L}{R}*(\frac{I}{1-I}+\frac{1-I}{1-I}) = \frac{L}{R}*(\frac{1}{1-I}) \rightarrow $$
$$ \frac{L}{R}*(\frac{1}{1-\frac{L*a}{R}}) $$

\subsection{B) Plot the total delay as a function of $\frac{L}{R}$.}
\begin{center}
  \includegraphics[width=0.5\textwidth]{q6_plot.png}
\end{center}

\pagebreak
\section{Q7}
\textbf{Consider the Figure below. Suppose that each link between the server and the client has a packet loss probability p, and the packet loss probabilities for these links are independent.}

\subsection{A) What is the probability that a packet (sent by the server) is successfully received by the receiver?}
The probability that a packet is succesfully recieved by the reciever is just the probability that all N links successfully recieved it. Therefore, it is simply the following:
$$ P_{success} = (1 - p)^N $$

\subsection{B) If a packet is lost in the path from the server to the client, then the server will re-transmit the packet. On average, how many times will the server re-transmit the packet in order for the client to successfully receive the packet?}
If a packet is lost in the path, the amount of retries that a packet has to be sent before it succeeds is the following:
$$ average\ retries = \frac{1}{(1 - p)^N} - 1 $$

\end{document}
